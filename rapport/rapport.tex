\documentclass{article}
%\usepackage[cyr]{aeguill}
\usepackage[utf8]{inputenc}
\usepackage[T1]{fontenc}
\usepackage[francais]{babel}
\usepackage{amsmath}
\usepackage{fancyhdr}
\usepackage{amsfonts}
\usepackage{makeidx}         
\title{Implémentation de l'algorithme du fast-marching pour la résolution d'un labyrinthe}
\usepackage[pdftex]{graphicx}
\author{Jean Prost \and Lucas Potin \and Edouard Gouteux}

\begin{document}
	
\section{L'algorithme du fast marching}

La méthode du fast marching a été introduite par James Setian. Cette méthode possède notamment des applications en mécanique des fluides et en traitement d'image. Cette méthode permet de résoudre l'équation d'Eikonal, de la forme :
\begin{equation}
|\nabla T|=\mathcal{F}
\end{equation}
Ici, $ \mathcal{F}$ et $T$ sont des fonctions $\mathbb{R}^n \to \mathbb{R}$, ou $n$ peut prendre la valeur 1, 2, où 3. $\mathcal{F}$ est la métrique donné du problème, et $T$ est la fonction à déterminer. Ici nous nous concentrerons sur des fonctions $ \mathcal{F}$ et $T$ de $\mathbb{R}^2 \to \mathbb{R}$. 
\section{Programmation}

\section{Résultats}


\end{document} 